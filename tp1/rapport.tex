\documentclass[10pt,a4paper]{article}
\usepackage[utf8]{inputenc}
\usepackage[T1]{fontenc}
\usepackage[french]{babel}
\usepackage{fancyhdr}
\pagestyle{fancy}
\usepackage{graphicx}

\renewcommand{\headrulewidth}{1pt}
\fancyhead[L]{\textbf{Mode d'emploi du }} 
\fancyhead[R]{Kevin Youna Rami Ben Mohamed Moncef Sekhsoukh}


\begin{document}
	\section{Explication des programmes}
	\subsection{programme.py}
	\paragraph{
		Le programme ./programme.py s'utilise avec 3 arguments : 
		\begin{description}
			\item Le fichier source c'est à dire le fichier depuis lequel on va charger la matrice sous format txt.
			\item Le sommet de d\'epart est le sommet pour lequel on aura la valeur 1 dans la liste mod\'elisant le vecteur.
			\item Le nombre de pas est le nombre d'it\'eration pour lequel on va faire le calcul $\ M^\top \times Vn $\ avec n \'egal au nombre de pas.
		\end{description}
		\begin{center} 
			\itshape{usage ./programme fichier_source sommet_depart nombre_de_pas}
		\end{center}}
	\subsection{programme2.py}
	\paragraph{
		Le programme ./programme2.py s'utilise avec 3 arguments:
		\begin{itemize}
			\item Le fichier source c'est à dire le fichier depuis lequel on va charger la matrice sous format txt
			\item Le sommet de d\'epart est le sommet pour lequel on aura la valeur 1 dans la liste mod\'elisant le vecteur
			\item Epsilon \'etant la valeur pour laquelle on va arr\'eter le calcul
		\end{itemize}
		\begin{center} 
			\textit{usage ./programme fichier_source sommet_depart epsilon}
		\end{center}}
	\subsection{programme3.py}
	\paragraph{
		Le programme ./programme3.py s'utilise avec 4 arguments:
		\begin{itemize}
			\item Le fichier source c'est à dire le fichier depuis lequel on va charger la matrice sous format txt
			\item Le sommet de d\'epart est le sommet pour lequel on aura la valeur 1 dans la liste mod\'elisant le vecteur
			\item Epsilon \'eant la valeur pour laquelle on va arr\'eter le calcul
			\item zap \'etant la valeur du facteur zap
		\end{itemize}
			\begin{center} 
				\textit{usage ./programme fichier_source sommet_depart epsilon zap}
			\end{center}}
\end{document}