\documentclass[10pt,a4paper]{article}
\usepackage[utf8]{inputenc}
\usepackage[T1]{fontenc}
\usepackage[french]{babel}
\usepackage{fancyhdr}
\pagestyle{fancy}
\usepackage{graphicx}

\renewcommand{\headrulewidth}{1pt}
\fancyhead[L]{\textbf{Mode d'emploi du }} 
\fancyhead[R]{Kevin Youna Rami Ben Mohamed Moncef Sekhsoukh}

\begin{document}
	\section{Explication des programmes}
	\subsection{programme.py}
	\begin{quote}[usage]
	\end{quote}
	\paragraph{
		Le programme ./programme.py s'utilise avec 3 arguments : }
		\begin{description}
			\item  Le fichier source c'est à dire le fichier depuis lequel on va charger la matrice sous format txt.
			\item  Le sommet de d\'epart est le sommet pour lequel on aura la valeur 1 dans la liste mod\'elisant le vecteur.
			\item  Le nombre de pas est le nombre d'it\'eration pour lequel on va faire le calcul $\ M^\top \times Vn $\ avec n \'egal au nombre de pas.
		\end{description}
\end{document}